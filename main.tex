\documentclass{beamer}

% Tema e estilo
\usetheme{Darmstadt}
\usefonttheme[onlylarge]{structurebold}
\setbeamerfont*{frametitle}{size=\normalsize,series=\bfseries}

% Idioma e codificação
\usepackage[utf8]{inputenc}
\usepackage[T1]{fontenc}
\usepackage[portuguese]{babel}

% Pacotes adicionais
\usepackage{graphicx}
\usepackage{times}
\usepackage[algoruled,longend]{algorithm2e}
\usepackage{tikz}
\usetikzlibrary{arrows}
\usepackage{ragged2e}
\usepackage{scalefnt}

% Estilo TikZ
\tikzstyle{block}=[draw opacity=0.7,line width=2.0cm]

% Remover ícones de navegação
\setbeamertemplate{navigation symbols}{}

\setbeamertemplate{footline}{
  \leavevmode
  \begin{beamercolorbox}[wd=\paperwidth,ht=0.25cm,dp=0.3cm,leftskip=0.5cm,rightskip=0.5cm]{author in head/foot}
    \hbox to \paperwidth{%
      \raisebox{-0.35cm}{\includegraphics[height=0.6cm]{Fig/cefet.png}}%
      \hfill
      \raisebox{-0.1cm}{\scriptsize Slide \insertframenumber{} de \inserttotalframenumber}%
      \hspace{0.9cm}%
    }%
  \end{beamercolorbox}
}


% Título e informações
\title[Inteligência Artificial]{Título do Trabalho}
\author[Tiago Alves de Oliveira]{Aluno \\Orientador: Nome do Orientador \\Coorientador: Nome do Coorientador}
\date{\today}

%--------------------------------------------------------------------------------------------------------------------------
\begin{document}

% Slide de título com imagem acima (sem usar figure desnecessariamente)
\begin{frame}[plain]
    \begin{center}
        \includegraphics[height=2cm]{Fig/cefetmg.jpg}
    \end{center}
    \titlepage
\end{frame}

%--------------------------------------------------------------------------------------------------------------------------
\section{Recomendações}
\begin{frame}{Orientações Importantes}
    \begin{itemize}
        \item Esse material foi elaborado tão somente como uma referência para a apresentação do projeto de TCC, podendo ser convenientemente adaptado pelo aluno e seu orientador;
        \item Sugere-se que a apresentação seja dimensionada pela relação de 1 slide para cada 1 minuto do tempo disponível;
        \item Evite excesso de texto, coloque tópicos para direcionar e lembrar o assunto mais importante que deseja abordar (observe fonte de letras e espaçamento);       
    \end{itemize}
\end{frame}

\begin{frame}{Orientações Importantes}
    \begin{itemize}
        \item Lembre-se que numa apresentação, o uso de figuras, quadros e tabelas favorece uma interpretação mais adequada;
        \item Finalmente, muito cuidado na expressão escrita e emprego das normas técnicas A apresentação oral deve durar de 20 minutos. Na sequência cada examinador poderá usar um tempo para avaliação.
    \end{itemize}
\end{frame}

\begin{frame}{Dicas para apresentação…}
    Para uma apresentação bem-sucedida, lembre-se sempre:
    \begin{itemize}
        \item Releia os slides - Passe, repasse e repita. Examine os títulos, os alinhamentos, as ilustrações. Não há nada torto ou deslocado? Algum texto ilegível? Talvez sobreposto? A ordem está correta? Todo slide tem um título enfatizando a sua idéia central? Após corrigir… releia mais uma vez!
        \item Ensaie - Pratique sua apresentação, sozinho a princípio, e depois para um pequeno grupo, se você puder. Peça para alguém se posicionar bem longe e avaliar se consegue ler seus slides. Use um relógio e cronômetro quanto tempo você passa em cada slide - e tente deslocar o ponto de equilíbrio, de modo a reservar mais tempo para os slides mais importantes no contexto da sua apresentação. Quando terminar de ensaiar, ensaie de novo!       
    \end{itemize}
\end{frame}

\begin{frame}{Dicas para apresentação…}
    Para uma apresentação bem-sucedida, lembre-se sempre:
    \begin{itemize}
        \item Atenção aos slides iniciais - É nos slides iniciais que você conquista ou joga fora a atenção da banca - começar um texto ou apresentação é uma arte em si. Capriche especialmente no visual deles, e se esforce para memorizá-los (bem como os demais fatos relacionados ao tema deles) de forma destacada. Vale a pena.
    \end{itemize}
\end{frame}

\begin{frame}{Cuidado com ....}
    \begin{itemize}
        \item Gestos e/ou movimento exagerados ou conflitantes.
        \item Monotonia.
        \item Excesso ou falta de formalidade: procure adequar o "tom" da apresentação ao público alvo.
        \item Uso inadequado ou mistura de registros e níveis lingüísticos (gíria, linguagem chula etc.).
        \item Uso do clichês, lugares comuns e frases feitas, a não ser que seja intencional.
    \end{itemize}
\end{frame}

\begin{frame}{Ainda...}
    \begin{itemize}
        \item Não se esqueça de que a banca já leu o seu trabalho, não é necessário e nem recomendável entrar em detalhes.
        \item Mantenha uma postura profissional diante da banca, tratando os professores com respeito.
        \item Cuidado para não ficar na frente da TELA enquanto fala, a banca tem que ler o que está escrito.
        \item Vista-se como se fosse a uma entrevista de emprego e não à praia.
    \end{itemize}
\end{frame}

\begin{frame}{Modelo dos Slides}
    \begin{itemize}
        \item Modelo dos slides
        \item Não esqueça de apagar essa seção e deixar somente a abaixo.
    \end{itemize}
\end{frame}

\section{Centro Federal de Educação Tecnológica de Minas Gerais - CEFET-MG Campus Divinópolis - Curso de Engenharia de Computação}

\begin{frame}{Sumário (opcional)}
    \small
    \begin{itemize}
        \item Introdução
        \item Objetivos
        \item Fundamentação Teórica
        \item Metodologia
        \item Resultados Esperados
        \item Conclusão
        \item Referências
    \end{itemize}
\end{frame}

\begin{frame}{Introdução}
    \small
    \begin{itemize}
        \item Apresentação do Tema / Posicionamento de Contexto
        \item Colocação objetiva do problema e apresentação da questão da pesquisa
        \item Justificativas / Motivação
        \item Para melhor detalhamento, “quebre” os assuntos em mais slides 
    \end{itemize}
\end{frame}

\begin{frame}{Objetivos}
    \small
    \begin{itemize}
        \item Objetivo Principal
        \begin{itemize}
            \item .....
        \end{itemize}
        \item Objetivos Específicos
        \begin{itemize}
            \item .....
            \item .....
        \end{itemize}
    \end{itemize}  
\end{frame}

\begin{frame}{Fundamentação Teórica ou Materiais e Métodos}
    \small
    \begin{enumerate}
        \item Caracterização do objeto de estudo
        \item Posicionamento do trabalho segundo seus fins (objetivos) e meios (procedimentos técnicos adotados)
        \item Planejamento da coleta de dados (universo e amostra, fontes, instrumentos e procedimentos)
        \item Planejamento da análise de dados
        \item Sempre que necessário, “quebre” os assuntos (sub-itens) em slides distintos
    \end{enumerate}
\end{frame}

\begin{frame}{Metodologia}
    \small
    \begin{itemize}
        \item Descrever aqui como é metodologia do seu trabalho
    \end{itemize}
\end{frame}

\begin{frame}{Resultados Esperados}
    \begin{itemize}
        \item Descrever os resultados que se espera com a pesquisa
        \item Use gráficos ou imagens de protótipo
        \item Apresente comparações com estudos anteriores ou benchmarks
    \end{itemize}
\end{frame}

\begin{frame}{Referências}
    \begin{itemize}
        \item Não é preciso  colocar aqui “todas” as referências, mas apenas aquelas consideradas fundamentais para o seu trabalho
        \item Use rigorosamente as normas da ABNT
    \end{itemize}
\end{frame}

\begin{frame}{Slide Opcional}
    \begin{itemize}
        \item Nesse slide você pode colocar uma palavra de agradecimento e um e-mail para contato.
    \end{itemize}
\end{frame}
\end{document}
